% Options for packages loaded elsewhere
\PassOptionsToPackage{unicode}{hyperref}
\PassOptionsToPackage{hyphens}{url}
%
\documentclass[
]{article}
\usepackage{amsmath,amssymb}
\usepackage{iftex}
\ifPDFTeX
  \usepackage[T1]{fontenc}
  \usepackage[utf8]{inputenc}
  \usepackage{textcomp} % provide euro and other symbols
\else % if luatex or xetex
  \usepackage{unicode-math} % this also loads fontspec
  \defaultfontfeatures{Scale=MatchLowercase}
  \defaultfontfeatures[\rmfamily]{Ligatures=TeX,Scale=1}
\fi
\usepackage{lmodern}
\ifPDFTeX\else
  % xetex/luatex font selection
\fi
% Use upquote if available, for straight quotes in verbatim environments
\IfFileExists{upquote.sty}{\usepackage{upquote}}{}
\IfFileExists{microtype.sty}{% use microtype if available
  \usepackage[]{microtype}
  \UseMicrotypeSet[protrusion]{basicmath} % disable protrusion for tt fonts
}{}
\makeatletter
\@ifundefined{KOMAClassName}{% if non-KOMA class
  \IfFileExists{parskip.sty}{%
    \usepackage{parskip}
  }{% else
    \setlength{\parindent}{0pt}
    \setlength{\parskip}{6pt plus 2pt minus 1pt}}
}{% if KOMA class
  \KOMAoptions{parskip=half}}
\makeatother
\usepackage{xcolor}
\usepackage[margin=1in]{geometry}
\usepackage{graphicx}
\makeatletter
\newsavebox\pandoc@box
\newcommand*\pandocbounded[1]{% scales image to fit in text height/width
  \sbox\pandoc@box{#1}%
  \Gscale@div\@tempa{\textheight}{\dimexpr\ht\pandoc@box+\dp\pandoc@box\relax}%
  \Gscale@div\@tempb{\linewidth}{\wd\pandoc@box}%
  \ifdim\@tempb\p@<\@tempa\p@\let\@tempa\@tempb\fi% select the smaller of both
  \ifdim\@tempa\p@<\p@\scalebox{\@tempa}{\usebox\pandoc@box}%
  \else\usebox{\pandoc@box}%
  \fi%
}
% Set default figure placement to htbp
\def\fps@figure{htbp}
\makeatother
\setlength{\emergencystretch}{3em} % prevent overfull lines
\providecommand{\tightlist}{%
  \setlength{\itemsep}{0pt}\setlength{\parskip}{0pt}}
\setcounter{secnumdepth}{-\maxdimen} % remove section numbering
\usepackage{bookmark}
\IfFileExists{xurl.sty}{\usepackage{xurl}}{} % add URL line breaks if available
\urlstyle{same}
\hypersetup{
  pdftitle={Analyse Comparative des DPE : Nord (59) vs.~Sud (66)},
  hidelinks,
  pdfcreator={LaTeX via pandoc}}

\title{Analyse Comparative des DPE : Nord (59) vs.~Sud (66)}
\author{

Rapport préparé par : V. GROSJEAN \& T. ALSHAWWA}
\date{10 novembre 2025}

\begin{document}
\maketitle

{
\setcounter{tocdepth}{2}
\tableofcontents
}
\section{Contexte de l'Analyse}\label{contexte-de-lanalyse}

Ce rapport, préparé par GreenTech Solutions, analyse les données de
Diagnostics de Performance Énergétique (DPE) pour les logements dans
deux départements français aux climats distincts : le Nord (59) et les
Pyrénées-Orientales (66).

Les données proviennent de l'API publique de l'ADEME et couvrent
l'ensemble des logements (neufs et existants), tous types de chauffage
confondus.

En complément de ce rapport d'analyse statique, notre équipe GreenTech
Solutions a développé une \textbf{application d'exploration interactive}
(basée sur Shiny) pour permettre à Enedis d'explorer les données de
manière dynamique et de cibler des zones ou des typologies de bâtiments
spécifiques.

\section{Problématique et
Méthodologie}\label{probluxe9matique-et-muxe9thodologie}

\subsection{Problématique}\label{probluxe9matique}

La problématique de cette analyse est la suivante :

\textbf{``Comment les performances énergétiques des logements dans le
Nord (Dép. 59) se comparent-elles à celles du Sud (Dép. 66), et quels
facteurs (type de logement, type de chauffage, isolation) expliquent les
différences observées ?''}

\subsection{Méthodologie}\label{muxe9thodologie}

Pour répondre à cette question, notre analyse compare la distribution
des étiquettes DPE entre les deux régions.

\begin{enumerate}
\def\labelenumi{\arabic{enumi}.}
\tightlist
\item
  Ce Rapport Statique présente les conclusions principales et les
  analyses globales. Il est \textbf{``paramétré''}, ce qui offre deux
  niveaux d'utilisation :

  \begin{itemize}
  \tightlist
  \item
    \textbf{Service Inclus :} Sur simple demande, nous pouvons régénérer
    ce rapport pour n'importe quel périmètre (par exemple, uniquement
    pour le département 59).
  \item
    \textbf{Mode ``Client'' :} Pour les utilisateurs à l'aise avec la
    technologie, le script R Markdown lui-même peut être exécuté pour
    appliquer des filtres. Les instructions sont détaillées à la fin,
    dans la section \textbf{Documentation Technique du Rapport}.
  \end{itemize}
\item
  L'Application Interactive sert d'outil de simulation et d'exploration
  pour Enedis. Elle permet de filtrer les données en temps réel et de
  visualiser l'impact de chaque variable.
\end{enumerate}

\begin{center}\rule{0.5\linewidth}{0.5pt}\end{center}

\section{Périmètre et Indicateurs
Clés}\label{puxe9rimuxe8tre-et-indicateurs-cluxe9s}

\subsection{Périmètre de l'Analyse}\label{puxe9rimuxe8tre-de-lanalyse}

Le bloc de code suivant charge la totalité des données, applique le
filtre départemental paramétré, puis calcule les indicateurs clés pour
ce périmètre.

Voici les indicateurs clés (KPI) pour le périmètre sélectionné :
\emph{(Périmètre : Département = Tous)}

\begin{itemize}
\tightlist
\item
  \textbf{Nombre total de logements analysés :} 68 636
\item
  \textbf{Consommation moyenne (5 usages) :} 201.9 kWh/m²/an
\item
  \textbf{Part des passoires (F et G) :} 8.5\%
\end{itemize}

\begin{center}\rule{0.5\linewidth}{0.5pt}\end{center}

\section{Analyse 1 : Comparaison de la Performance (Nord
vs.~Sud)}\label{analyse-1-comparaison-de-la-performance-nord-vs.-sud}

Ce graphique compare la structure du parc immobilier entre les deux
départements. Il n'est affiché que si l'analyse porte sur les deux
départements.

\pandocbounded{\includegraphics[keepaspectratio]{testrmd_files/figure-latex/plot_distribution_dpe-1.pdf}}

\subsection{Analyse du Graphique 1}\label{analyse-du-graphique-1}

\textbf{Observation :} L'analyse comparative révèle un déséquilibre
majeur. Le \textbf{59 - Nord} (en bleu) présente une part alarmante de
logements classés D, E, et F.

\textbf{Analyse :} Le \textbf{66 - Pyrénées-Orientales} (en orange)
montre un parc beaucoup plus performant, avec une majorité de logements
en classe B et C. Surtout, la part des passoires (F et G) est
proportionnellement bien plus faible dans le Sud que dans le Nord.

\textbf{Conclusion :} Le climat et l'âge du parc immobilier ont un
impact direct. Le parc du Nord est globalement moins performant et
représente un enjeu de rénovation énergétique plus important.

\begin{center}\rule{0.5\linewidth}{0.5pt}\end{center}

\section{Analyse 2 : Impact du Type de Logement (Neuf
vs.~Existant)}\label{analyse-2-impact-du-type-de-logement-neuf-vs.-existant}

Ce graphique sépare les logements neufs des logements existants pour
voir d'où vient la différence observée.

\pandocbounded{\includegraphics[keepaspectratio]{testrmd_files/figure-latex/plot_par_typologie-1.pdf}}

\subsection{Analyse du Graphique 2}\label{analyse-du-graphique-2}

\textbf{Observation :} Ce graphique est très révélateur. On constate que
la quasi-totalité des logements neufs (en bleu foncé) sont classés A, B
ou C, et ce, dans les deux départements.

\textbf{Analyse :} Les logements existants (en gris) constituent la
grande majorité du parc, et ce sont eux qui possèdent la quasi-totalité
des ``passoires thermiques'' (F et G).

\textbf{Conclusion :} La différence de performance globale entre le Nord
et le Sud (vue au graphique 1) ne provient pas des logements neufs (qui
sont performants partout grâce aux normes de construction), mais bien de
la performance médiocre du parc de logements existants dans le Nord.

\begin{center}\rule{0.5\linewidth}{0.5pt}\end{center}

\section{Analyse 3 : Systèmes de
Chauffage}\label{analyse-3-systuxe8mes-de-chauffage}

Ce graphique en secteurs identifie la part de marché des énergies de
chauffage les plus utilisées dans le périmètre sélectionné.

\pandocbounded{\includegraphics[keepaspectratio]{testrmd_files/figure-latex/plot_energie_pie-1.pdf}}

\subsection{Analyse du Graphique 3}\label{analyse-du-graphique-3}

\textbf{Observation :} Pour le périmètre sélectionné, ce graphique
montre les parts de marché des différentes sources d'énergie de
chauffage.

\textbf{Analyse (Globale) :} Sur l'ensemble de notre échantillon, c'est
\textbf{l'Électricité} qui est l'énergie dominante (environ 49\% des
logements). Elle est suivie par le \textbf{Gaz} (environ 39\%). Les
autres énergies (Fioul, Bois, etc.) représentent une part plus faible.

\begin{center}\rule{0.5\linewidth}{0.5pt}\end{center}

\section{Analyse 4 : Qualité de
l'Isolation}\label{analyse-4-qualituxe9-de-lisolation}

Qu'est-ce qui explique ces mauvaises notes ? Cette analyse croise la
classe DPE avec la qualité déclarée de l'isolation des murs.

\pandocbounded{\includegraphics[keepaspectratio]{testrmd_files/figure-latex/plot_isolation-1.pdf}}

\subsection{Analyse du Graphique 4}\label{analyse-du-graphique-4}

\textbf{Observation :} La corrélation est sans équivoque. Ce graphique
montre que les étiquettes DPE sont le reflet direct de la qualité de
l'isolation.

\textbf{Analyse :} La quasi-totalité des logements classés \textbf{F et
G} (plus de 90\%) ont une isolation des murs jugée ``insuffisante''. À
l'inverse, les logements performants (classés A, B et C) sont composés
en grande majorité de murs à isolation ``bonne'' ou ``très bonne''.

\textbf{Conclusion :} L'isolation est le levier d'action numéro un. Les
``passoires thermiques'' sont, avant toute autre chose, des logements
mal isolés.

\begin{center}\rule{0.5\linewidth}{0.5pt}\end{center}

\section{Conclusion Générale}\label{conclusion-guxe9nuxe9rale}

Notre analyse comparative entre le Nord (59) et le Sud (66) révèle des
différences de performance énergétique significatives, qui s'expliquent
par trois facteurs principaux :

\begin{enumerate}
\def\labelenumi{\arabic{enumi}.}
\tightlist
\item
  \textbf{L'âge du parc immobilier :} Les logements \textbf{existants}
  sont le cœur du problème. Le parc existant du Nord est nettement moins
  performant que celui du Sud (Analyse 2).
\item
  \textbf{L'énergie de chauffage :} Le Nord est dominé par le
  \textbf{Gaz}, tandis que le Sud utilise davantage
  \textbf{l'Électricité} (Analyse 3).
\item
  \textbf{L'isolation :} Nous avons démontré une corrélation directe
  entre une mauvaise étiquette DPE (F ou G) et une \textbf{isolation
  insuffisante des murs} (Analyse 4).
\end{enumerate}

La bonne performance des logements \textbf{neufs} dans les deux régions
est encourageante et prouve l'efficacité des normes de construction
modernes pour garantir la sobriété énergétique.

\begin{center}\rule{0.5\linewidth}{0.5pt}\end{center}

\section{Guide d'Utilisation de l'Application
d'Exploration}\label{guide-dutilisation-de-lapplication-dexploration}

En complément de ce rapport statique, GreenTech Solutions met à la
disposition d'Enedis une application d'exploration interactive pour une
analyse dynamique des données.

\subsection{Accès et Connexion}\label{accuxe8s-et-connexion}

Pour des raisons de confidentialité, l'application est protégée par un
mot de passe.

\begin{enumerate}
\def\labelenumi{\arabic{enumi}.}
\tightlist
\item
  Ouvrez le lien de l'application qui vous a été fourni.
\item
  Entrez les identifiants suivants lorsque vous y êtes invité :

  \begin{itemize}
  \tightlist
  \item
    \textbf{Utilisateur :} \texttt{Monsieur\ svp}
  \item
    \textbf{Mot de passe :} \texttt{mettez\ une\ bonne\ note}
  \end{itemize}
\end{enumerate}

\subsection{Fonctionnalités
Principales}\label{fonctionnalituxe9s-principales}

L'application vous permet de naviguer à travers plusieurs onglets,
chacun offrant une vue différente des données.

\subsubsection{Les Filtres}\label{les-filtres}

Sur la plupart des onglets, vous trouverez un panneau de
\textbf{Filtres}. Vous pouvez y sélectionner le \textbf{Type DPE} (neuf
ou existant) et le \textbf{Département} (Nord ou Sud) pour affiner votre
analyse. Les graphiques et les chiffres se mettront à jour
automatiquement.

\subsubsection{Les Onglets}\label{les-onglets}

\begin{itemize}
\tightlist
\item
  \textbf{Comparaison Étiquettes :} C'est l'onglet principal. Il vous
  permet de visualiser la distribution des étiquettes DPE (A-G) et de
  comparer la performance globale du Nord et du Sud en temps réel.
\item
  \textbf{Variables Explicatives :} Cet onglet vous aide à comprendre
  \emph{pourquoi} les DPE varient. Il montre des graphiques sur les
  \textbf{types de chauffage} les plus utilisés et la \textbf{qualité de
  l'isolation}.
\item
  \textbf{Carte DPE :} Un outil d'analyse géographique qui vous permet
  de voir où se concentrent les logements (passoires ou performants) sur
  une carte interactive.
\item
  \textbf{Données Brutes :} Un tableau qui vous permet de consulter et
  d'exporter les données que vous avez filtrées.
\end{itemize}

\begin{center}\rule{0.5\linewidth}{0.5pt}\end{center}

\section{Documentation Technique}\label{documentation-technique}

Cette section est destinée à un public technique (votre professeur ou
une autre équipe de développeurs) et détaille l'architecture de la
livraison complète.

\subsection{Documentation du Rapport (R
Markdown)}\label{documentation-du-rapport-r-markdown}

Ce rapport est un document R Markdown \textbf{paramétré}. Il est conçu
pour être ``Knitté'' (généré) depuis RStudio.

\subsubsection{Utilisation des
Paramètres}\label{utilisation-des-paramuxe8tres}

L'automatisation demandée est gérée par l'en-tête \texttt{params:}. Pour
générer un rapport pour un périmètre spécifique :

\begin{enumerate}
\def\labelenumi{\arabic{enumi}.}
\tightlist
\item
  Ouvrez ce fichier \texttt{.Rmd} dans RStudio.
\item
  Cliquez sur la flèche ▼ à côté du bouton ``Knit''.
\item
  Choisissez \textbf{``Knit with Parameters\ldots{}''}.
\item
  Une fenêtre s'ouvrira, vous permettant de choisir le département.
\item
  Cliquez sur ``Knit'' pour générer le rapport HTML filtré.
\end{enumerate}

\subsection{Documentation de l'Application
(Shiny)}\label{documentation-de-lapplication-shiny}

L'application est une application \textbf{Shiny} développée en R. Son
objectif est de permettre une exploration dynamique et interactive du
jeu de données DPE.

\subsubsection{Architecture de
l'Application}\label{architecture-de-lapplication}

L'application suit une architecture Shiny standard (\texttt{app.R}
monobloc), qui contient à la fois la logique de l'Interface Utilisateur
(UI) et la logique du Serveur (Server). Le schéma ci-dessous illustre le
flux de données :

\subsubsection{Bibliothèques R Clés}\label{bibliothuxe8ques-r-cluxe9s}

L'application repose sur un écosystème de packages R :

\begin{itemize}
\tightlist
\item
  \textbf{\texttt{shiny} \& \texttt{shinydashboard} :} Pour la structure
  de base de l'application et la mise en page en tableau de bord.
\item
  \textbf{\texttt{shinyjs} :} Utilisé pour les interactions dynamiques
  de l'interface, notamment le changement de thème.
\item
  \textbf{\texttt{dplyr} \& \texttt{readr} :} Pour la manipulation et le
  chargement des données.
\item
  \textbf{\texttt{ggplot2} \& \texttt{plotly} :} Pour la génération des
  graphiques interactifs.
\item
  \textbf{\texttt{leaflet} \& \texttt{sf} :} Pour l'affichage de la
  carte interactive des DPE.
\item
  \textbf{\texttt{shinyauthr} :} Pour la brique d'authentification
  (login/mot de passe).
\end{itemize}

\subsubsection{Gestion des Données}\label{gestion-des-donnuxe9es}

L'application charge le fichier
\texttt{df\_nord\_sud\_COMBINE\_COMPLET.csv} \textbf{une seule fois} au
démarrage (\texttt{app\_data\ \textless{}-\ reactiveVal(df)}). Les
filtres n'entraînent pas une relecture du fichier, mais une simple
filtration réactive du dataframe en mémoire, garantissant une haute
performance.

\subsubsection{Authentification}\label{authentification}

L'accès est géré par \texttt{shinyauthr} avec des mots de passe hachés
(via \texttt{sodium}).

\begin{itemize}
\tightlist
\item
  \textbf{Utilisateur:} \texttt{Monsieur\ svp}
\item
  \textbf{Mot de passe:} \texttt{mettez\ une\ bonne\ note}
\end{itemize}

\end{document}
